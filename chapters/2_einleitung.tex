\chapter{Introduction}
%Ziel des Versuchs.
Aim of this experiment is to examin the statistical characteristics of radioactive decay as well as comparing the expected
with the actual behavior of the core instrument - the \textsc{Geiger-Müller}-Tube (GMT).\par
A GMT is a device to detect and quantify radiation emmitted from a given direction by counting ionization events inside
its volume. The nature of the GMT does not allow to distinguish between types of sources of ionization i.e. \(\alpha\)-,
\(\beta\)- or \(\gamma\)-radiation.\par\medskip
Radioactive decay is a statistical process. According to quantum physics it is impossible to predict the life span of a
single atom. Given a significantly large number of atoms one can state the overall time until half of the nuclei did
disintegrate (\cref{eq:halflife}) with the for each radioactive nuclide a chracteristical mean lifespan \(\tau\).
The amount of non-decayed radio-nuclides after a time \(t\) is given by \cref{eq:zerfallsgesetz}.
\begin{equation}
    N(t) = N_0 e^{\nicefrac{-1}{\tau}}
    \label{eq:zerfallsgesetz}
\end{equation}
\begin{equation}
    T_{\nicefrac{1}{2}} = \frac{\ln2}{\tau}
    \label{eq:halflife}
\end{equation}